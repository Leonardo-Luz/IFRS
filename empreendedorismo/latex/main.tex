\documentclass{article}
\usepackage[utf8]{inputenc}
\usepackage[brazil]{babel}
\usepackage{multirow}
\usepackage{array}
\usepackage[table]{xcolor}
\usepackage{geometry}
\usepackage{float}
\usepackage{xparse}

\geometry{paperwidth=700pt,paperheight=1300pt, margin=1cm}
\definecolor{lightgray}{gray}{0.9}
\definecolor{blueval}{RGB}{0,0,180} % azul para terceira validação
\definecolor{greenval}{RGB}{0,150,0} % verde para quarta validação
\newcommand{\hoje}{08/10/2025}

\newcommand{\canvasTitle}{}
\newcommand{\canvasDescription}{}
\newcommand{\canvasCaption}{}
\newcommand{\canvasPrimeiroBloco}{}
\newcommand{\canvasSegundoBloco}{}
\newcommand{\canvasTerceiroBloco}{}
\newcommand{\canvasQuartoBloco}{}

\newcommand{\setcanvas}[7]{%
  \def\canvasTitle{#1}%
  \def\canvasDescription{#2}%
  \def\canvasCaption{#3}%
  \def\canvasPrimeiroBloco{#4}%
  \def\canvasSegundoBloco{#5}%
  \def\canvasTerceiroBloco{#6}%
  \def\canvasQuartoBloco{#7}%
}

\newcommand{\appendto}[2]{%
  \expandafter\gdef\expandafter#1\expandafter{#1 #2}%
}

\newcommand{\canvas}[1]{%
  \renewcommand{\arraystretch}{1.8}
  \section*{\canvasTitle}
  \canvasDescription
  \vspace{0.5cm}

  \begin{table}[H]
  \centering
  \caption{\canvasCaption}
  \setlength{\tabcolsep}{4pt}
  \begin{tabular}{|p{6cm}|p{6cm}|p{6cm}|}
  \hline
  \rowcolor{lightgray}
  \textbf{Parceiros Chave} & \centering\textbf{PROPOSTA DE VALOR} & \textbf{Relacionamento com Clientes} \\
  \cline{1-3}
  \canvasPrimeiroBloco
  \hline
  \rowcolor{lightgray}
  \textbf{Atividades Chave} & & \textbf{Canais} \\
  \cline{1-3}
  \canvasSegundoBloco
  \hline
  \rowcolor{lightgray}
  \textbf{Recursos Chave} & & \textbf{Segmentos de Clientes} \\
  \cline{1-3}
  \canvasTerceiroBloco
  \hline
  \rowcolor{lightgray}
  \textbf{Estrutura de Custos} & & \textbf{Fluxo de Receitas} \\
  \hline
  \canvasQuartoBloco
  \hline
  \end{tabular}

  \begin{center}
  \ifx#1\empty\else
  \textbf{Legenda:} \\
  #1
  \fi
  \end{center}
  \end{table}
}

\begin{document}

% ---------- HEADER ----------
\begin{table}[H]
\centering
\renewcommand{\arraystretch}{1.3}
\setlength{\tabcolsep}{10pt}
\rowcolors{1}{lightgray}{white}
\begin{tabular}{|p{4cm}|p{7cm}|p{3cm}|p{2cm}|}
\hline
\textbf{Projeto} & \textbf{Criado Por} & \textbf{Data} & \textbf{Versão} \\
\hline
ColabHUB & Daniel, Diego, Gustavo Martins, Gustavo Policarpo e Leonardo & \hoje & 1.0 \\
\hline
\end{tabular}
\end{table}

\section*{Canvas Inicial}
O projeto \textbf{ColabHUB} tem como proposta conectar desenvolvedores, empreendedores e empresas em um ambiente colaborativo, onde ideias podem se transformar em projetos reais. Esta primeira versão do Canvas apresenta a visão inicial do grupo, ainda sem validações externas.

\vspace{0.5cm}

% ---------- CANVAS 1: Conteúdo Original ----------
\begin{table}[H]
\centering
\caption{Modelo de negócio de uma plataforma para colaboração em projetos para desenvolvedores e entusiastas}
\renewcommand{\arraystretch}{1.8}
\setlength{\tabcolsep}{4pt}
\begin{tabular}{|p{6cm}|p{6cm}|p{6cm}|}
\hline
\rowcolor{lightgray}
\textbf{Parceiros Chave} &
\centering\textbf{PROPOSTA DE VALOR} &
\textbf{Relacionamento com Clientes} \\
\cline{1-1}\cline{2-2}\cline{3-3}
• Empresas de Tecnologia & • Networking & • Website \\
• Provedores de Hospedagem & • Conveniência & • Redes Sociais \\
& • Colaboração & • Aplicativo Móvel \\
& • Estudo & \\
\hline
\rowcolor{lightgray}
\textbf{Atividades Chave} & & \textbf{Canais} \\
\cline{1-1}\cline{2-2}\cline{3-3}
• Desenvolvimento da Plataforma & & • Google Adwords \\
• Promoção da Plataforma nos Canais & & • Facebook Ads \\
• Busca por Parceiros & & • Email Marketing \\
& & • Afiliados (Influenciadores) \\
& & • Social Orgânico \\
& & • Blog \\
\hline
\rowcolor{lightgray}
\textbf{Recursos Chave} & & \textbf{Segmentos de Clientes} \\
\cline{1-1}\cline{2-2}\cline{3-3}
• Aquisição de Equipamentos & & • Desenvolvedores \\
• Equipe de desenvolvimento & & • UI/UX Designers \\
• Contratação de Servidor & & • Empreendedores \\
• Compra de Domínio & & • Startups \\
• Verba para Marketing & & \\
\hline
\rowcolor{lightgray}
\textbf{Estrutura de Custos} & & \textbf{Fluxo de Receitas} \\
\hline
• Hospedagem & & • Planos de Assinatura \\
• Domínio & & • Publicidade de Parceiros \\
• Marketing & & • Patrocinadores \\
• Manutenção & & \\
• Serviços Externos (API de pagamentos, integração com repositórios) & & \\
\hline
\end{tabular}
\end{table}


\newpage

% ---------- CANVAS 2: Conteúdo Original + Especialista (Vermelho) ----------

\section*{Primeira Validação}
Após uma conversa com um especialista técnico, foram identificados pontos de melhoria voltados à parte operacional e estrutural do projeto. Essa validação ajudou a ajustar os recursos e atividades principais para garantir a viabilidade do ColabHUB.

\vspace{0.5cm}

\begin{table}[H]
\centering
\caption{Modelo de negócio de uma plataforma para colaboração em projetos para desenvolvedores e entusiastas}
\renewcommand{\arraystretch}{1.8}
\setlength{\tabcolsep}{4pt}
\begin{tabular}{|p{6cm}|p{6cm}|p{6cm}|}
\hline
\rowcolor{lightgray}
\textbf{Parceiros Chave} &
\centering\textbf{PROPOSTA DE VALOR} &
\textbf{Relacionamento com Clientes} \\
\cline{1-1}\cline{2-2}\cline{3-3}
• Empresas de Tecnologia & • Networking & • Website \\
• Provedores de Hospedagem & • Conveniência & • Redes Sociais \\
& • Colaboração & • Aplicativo Móvel \\
& • Estudo & \\
\textcolor{red}{• Instituições de ensino.} &
\textcolor{red}{• Simular o mercado de trabalho real.} &
\textcolor{red}{• Fórum interno para se comunicar dentro do projeto.} \\
& \textcolor{red}{• Criação de Portfólio} &
\\
\hline
\rowcolor{lightgray}
\textbf{Atividades Chave} & & \textbf{Canais} \\
\cline{1-1}\cline{2-2}\cline{3-3}
• Desenvolvimento da Plataforma & & • Google Adwords \\
• Promoção da Plataforma nos Canais & & • Facebook Ads \\
• Busca por Parceiros & & • Email Marketing \\
& & • Afiliados (Influenciadores) \\
& & • Social Orgânico \\
& & • Blog \\
& & 
\textcolor{red}{• Divulgar em lugares que desenvolvedores realmente usam como Linkedin e Discord.} \\
\hline
\rowcolor{lightgray}
\textbf{Recursos Chave} & & \textbf{Segmentos de Clientes} \\
\cline{1-1}\cline{2-2}\cline{3-3}
• Aquisição de Equipamentos & & • Desenvolvedores \\
• Equipe de desenvolvimento & & • UI/UX Designers \\
• Contratação de Servidor & & • Empreendedores \\
• Compra de Domínio & & • Startups \\
• Verba para Marketing & & \\
\textcolor{red}{• Coordenador para orientar e direcionar o marketing} &
& 
\textcolor{red}{• Behance, plataforma para encontrar portfólios} \\
\textcolor{red}{• Iniciar utilizando serviços gratuitos ou baratos, como Vercel} &
& 
\textcolor{red}{• Começar focando em desenvolvedores por já haver um conexões com esta área.} \\
\hline
\rowcolor{lightgray}
\textbf{Estrutura de Custos} & & \textbf{Fluxo de Receitas} \\
\hline
• Hospedagem & & • Planos de Assinatura \\
• Domínio & & • Publicidade de Parceiros \\
• Marketing & & • Patrocinadores \\
• Manutenção & & \\
• Serviços Externos (API de pagamentos, integração com repositórios) & & \\
\textcolor{red}{• Monday, plataforma para gerenciar custos e organizar o fluxo de trabalho.} &
&
\textcolor{red}{• Consultoria de especialistas da área para auxiliar no projeto.} \\
\hline
\end{tabular}

\begin{center}
\textbf{Legenda:} \\
\textcolor{red}{\textbf{Vermelho}} = Primeira validação
\end{center}

\end{table}

\newpage

% ---------- CANVAS 3: Preto + Vermelho + Azul (Empresário) ----------
\section*{Segunda Validação}
Após a análise de um especialista com uma visão mais comercial, foram feitas melhorias no posicionamento do ColabHUB, focando em estratégias de mercado, atração de público e sustentabilidade financeira.

\vspace{0.5cm}

\begin{table}[H]
\centering
\caption{modelo de negócio de uma plataforma para colaboração em projetos para desenvolvedores e entusiastas}
\renewcommand{\arraystretch}{1.8}
\setlength{\tabcolsep}{4pt}
\begin{tabular}{|p{6cm}|p{6cm}|p{6cm}|}
\hline
\rowcolor{lightgray}
\textbf{parceiros chave} &
\centering\textbf{proposta de valor} &
\textbf{relacionamento com clientes} \\
\cline{1-1}\cline{2-2}\cline{3-3}
• empresas de tecnologia & • networking & • website \\
• provedores de hospedagem & • conveniência & • redes sociais \\
& • colaboração & • aplicativo móvel \\
& • estudo & \\
\textcolor{red}{• Instituições de ensino.} &
\textcolor{red}{• Simular o mercado de trabalho real.} &
\textcolor{red}{• Fórum interno para se comunicar dentro do projeto.} \\
& \textcolor{red}{• criação de portfólio} & \\
\textcolor{blueval}{• Plataforma de contratação.} &
\textcolor{blueval}{• Fomentar o trabalho em equipe.} &
\textcolor{blueval}{• Criar formas de manter contato frequente e útil com usuários.} \\
\textcolor{blueval}{• priorizar parcerias com empresas de tecnologia da região} & & \\
\hline
\rowcolor{lightgray}
\textbf{atividades chave} & & \textbf{canais} \\
\cline{1-1}\cline{2-2}\cline{3-3}
• desenvolvimento da plataforma & & • google adwords \\
• promoção da plataforma nos canais & & • facebook ads \\
• busca por parceiros & & • email marketing \\
& & • afiliados (influenciadores) \\
& & • social orgânico \\
& & • blog \\
& & \textcolor{red}{• divulgar em lugares que desenvolvedores realmente usam como linkedin e discord.} \\
\textcolor{blueval}{• Chat para comunicação interna.} &
& 
\textcolor{blueval}{• Comunidade do Whatsapp} \\
\hline
\rowcolor{lightgray}
\textbf{recursos chave} & & \textbf{segmentos de clientes} \\
\cline{1-1}\cline{2-2}\cline{3-3}
• aquisição de equipamentos & & • desenvolvedores \\
• equipe de desenvolvimento & & • ui/ux designers \\
• contratação de servidor & & • empreendedores \\
• compra de domínio & & • startups \\
• verba para marketing & & \\
\textcolor{red}{• Coordenador para orientar e direcionar o marketing} &
& 
\textcolor{red}{• Behance, plataforma para encontrar portfólios} \\
\textcolor{red}{• Iniciar utilizando serviços gratuitos ou baratos, como vercel} &
& 
\textcolor{red}{• começar focando em desenvolvedores por já haver um conexões com esta área.} \\
\textcolor{blueval}{• Contratação de gerente para orientar o crescimento da empresa.} &
& 
\textcolor{blueval}{• Estudantes} \\
\hline
\rowcolor{lightgray}
\textbf{estrutura de custos} & & \textbf{fluxo de receitas} \\
\hline
• hospedagem & & • planos de assinatura \\
• domínio & & • publicidade de parceiros \\
• \textcolor{blueval}{priorizar o} marketing \textcolor{blueval}{apenas quando o projeto estiver estável} & & • patrocinadores \\
• manutenção & & \\
• serviços externos (api de pagamentos, integração com repositórios) & & \\
\textcolor{red}{• Monday, plataforma para gerenciar custos e organizar o fluxo de trabalho.} &
&
\textcolor{red}{• Consultoria de especialistas da área para auxiliar no projeto.} \\
\textcolor{blueval}{• Métricas e acompanhamento.} &
&
\textcolor{blueval}{• Assinatura.} \\
& & \textcolor{blueval}{• limitar a quantidade de colaboradores em planos gratuitos}  \\
\hline
\end{tabular}

\begin{center}
\textbf{Legenda:} \\
\textcolor{red}{\textbf{Vermelho}} = Primeira validação
\textcolor{blueval}{\textbf{Azul}} = Segunda validação
\end{center}

\end{table}

\newpage

% ---------- CANVAS 4: Preto + Vermelho + Azul + Verde (Pesquisa de Usuários) ----------
\section*{Terceira Validação}
Após a aplicação de uma pesquisa com o público-alvo, confirmou-se que o ColabHUB é um projeto relevante. Esta etapa consolidou as ideias anteriores e direcionou os próximos passos de acordo com o que o público mais valorizou.

\vspace{0.5cm}

\begin{table}[H]
\centering
\caption{Modelo de negócio de uma plataforma para colaboração em projetos para desenvolvedores e entusiastas}
\renewcommand{\arraystretch}{1.8}
\setlength{\tabcolsep}{4pt}
\begin{tabular}{|p{6cm}|p{6cm}|p{6cm}|}
\hline
\rowcolor{lightgray}
\textbf{Parceiros Chave} &
\centering\textbf{PROPOSTA DE VALOR} &
\textbf{Relacionamento com Clientes} \\
\cline{1-1}\cline{2-2}\cline{3-3}
• Empresas de Tecnologia & • Networking & • Website \\
• Provedores de Hospedagem & • Conveniência & • Redes Sociais \\
& • Colaboração & • Aplicativo Móvel \\
& • Estudo & \\
\textcolor{red}{• Instituições de ensino.} &
\textcolor{red}{• Simular o mercado de trabalho real.} &
\textcolor{red}{• Fórum interno para se comunicar dentro do projeto.} \\
& \textcolor{red}{• Criação de Portfólio} & \\
\textcolor{blueval}{• Plataforma de contratação.} &
\textcolor{blueval}{• Fomentar o trabalho em equipe.} &
\textcolor{blueval}{• Criar formas de manter contato frequente e útil com usuários.} \\
\textcolor{blueval}{• Priorizar parcerias com empresas de tecnologia da região} & & \\
\hline
\rowcolor{lightgray}
\textbf{Atividades Chave} & & \textbf{Canais} \\
\cline{1-1}\cline{2-2}\cline{3-3}
• Desenvolvimento da Plataforma & & • Google Adwords \\
• Promoção da Plataforma nos Canais & & • Facebook Ads \\
• Busca por Parceiros & & • Email Marketing \\
& & • Afiliados (Influenciadores) \\
& & • Social Orgânico \\
& & • Blog \\
& & \textcolor{red}{• Divulgar em lugares que desenvolvedores realmente usam como Linkedin e Discord.} \\
\textcolor{blueval}{• Chat para comunicação interna.} &
& 
\textcolor{blueval}{• Comunidade do Whatsapp} \\
\hline
\rowcolor{lightgray}
\textbf{Recursos Chave} & & \textbf{Segmentos de Clientes} \\
\cline{1-1}\cline{2-2}\cline{3-3}
• Aquisição de Equipamentos & & • Desenvolvedores \\
• Equipe de desenvolvimento & & • UI/UX Designers \\
• Contratação de Servidor & & • Empreendedores \\
• Compra de Domínio & & • Startups \\
• Verba para Marketing & & \\
\textcolor{red}{• Coordenador para orientar e direcionar o marketing} &
& 
\textcolor{red}{• Behance, plataforma para encontrar portfólios} \\
\textcolor{red}{• Iniciar utilizando serviços gratuitos ou baratos, como Vercel} & & 
\textcolor{red}{• Começar focando em desenvolvedores por já haver um conexões com esta área.} \\
\textcolor{blueval}{• Contratação de gerente para orientar o crescimento da empresa.} &
& 
\textcolor{blueval}{• Estudantes} \\
\hline
\rowcolor{lightgray}
\textbf{Estrutura de Custos} & & \textbf{Fluxo de Receitas} \\
\hline
• Hospedagem & & • Planos de Assinatura \\
• Domínio & & • Publicidade de Parceiros \\
• \textcolor{blueval}{Priorizar o} Marketing \textcolor{blueval}{apenas quando o projeto estiver estável} & & • Patrocinadores \\
• Manutenção & & \\
• Serviços Externos (API de pagamentos, integração com repositórios) & & \\
\textcolor{red}{• Monday, plataforma para gerenciar custos e organizar o fluxo de trabalho.} &
&
\textcolor{red}{• Consultoria de especialistas da área para auxiliar no projeto.} \\
\textcolor{blueval}{• Métricas e acompanhamento.} &
&
\textcolor{blueval}{• Assinatura.} \\
& & \textcolor{blueval}{• Limitar a quantidade de colaboradores em planos gratuitos}  \\
\hline
\end{tabular}

\begin{center}
\textbf{Legenda:} \\
\textcolor{red}{\textbf{Vermelho}} = Primeira validação \quad
\textcolor{blueval}{\textbf{Azul}} = Segunda validação \quad
\textcolor{greenval}{\textbf{Verde}} = Terceira validação
\end{center}

\end{table}

\end{document}
